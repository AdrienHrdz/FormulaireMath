\documentclass{article}
\usepackage[utf8]{inputenc}

% Language setting
% Replace `english' with e.g. `spanish' to change the document language
\usepackage[french]{babel}

% Set page size and margins
% Replace `letterpaper' with `a4paper' for UK/EU standard size
\usepackage[letterpaper,top=2cm,bottom=2cm,left=3cm,right=3cm,marginparwidth=1.75cm]{geometry}

% Useful packages
\usepackage{amsmath}
\usepackage{graphicx}
\usepackage[colorlinks=true, allcolors=blue]{hyperref}
\usepackage{float}
\usepackage{wrapfig}
\usepackage[T1]{fontenc}
\usepackage{comment}
\usepackage{authblk}
\usepackage{tikz}
\usepackage{pgfplots}
\pgfplotsset{compat=1.15}
\usepackage{mathrsfs}
\usetikzlibrary{arrows}
\usepackage{caption}
\usepackage{subcaption}
%\usepackage{biblatex}
%\addbibresource{biblio.bib}
\usepackage{csquotes}
\usepackage{multicol}
\newtheorem{theorem}{Theorem}

\title{Formulaire de mathématiques}
\author{Adrien Hernandez}
\date{}

\begin{document}

\maketitle
\section{Analyse}
\subsection{Sommes et Séries}
\begin{theorem}[Pi et Fibonacci]
    $$ \frac{\pi}{4} = \sum_{k=1}^{n} \arctan\left(\frac{1}{F_{2k+1}}\right) + \arctan\left(\frac{1}{F_{2n+2}}\right)$$
\end{theorem}
Ref : \href{https://www.youtube.com/watch?v=nZRKNth6OSA}{Video Axel Arno}

Lorsqu'on fait tendre $n$ vers l'infini, on obtient la formule suivante : 
$$ \frac{\pi}{4} = \sum_{k=1}^{\infty} \arctan\left(\frac{1}{F_{2k+1}}\right) $$
\subsection{Calcul différentiel}

\subsection{Calcul intégral}
\begin{theorem}[King's property]

    $$ \int_a^b f(x) \mathrm{d}x = \int_b^a f(a+b-x) \mathrm{d}x $$
    
\end{theorem}

\section{Algèbre}

\section{Complexes}
\begin{theorem}[De Moivre]
    $$(\cos(\theta) + \sin(\theta))^n =  \cos(n\theta) + \sin(n\theta)$$
\end{theorem}

\section{Probabilités}

$$ \left\langle \varphi  \vert  \hat{A} \vert  \psi\right\rangle  $$
\end{document}